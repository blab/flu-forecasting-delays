%%%%%%%%%%%%%%%%%%%%%%%%%%%%%%%%%%%%%%%%%%%%%%%%%%%%%%%%%%%%
%%% ELIFE ARTICLE TEMPLATE
%%%%%%%%%%%%%%%%%%%%%%%%%%%%%%%%%%%%%%%%%%%%%%%%%%%%%%%%%%%%
%%% PREAMBLE
\documentclass[9pt,lineno]{elife}
% Use the onehalfspacing option for 1.5 line spacing
% Use the doublespacing option for 2.0 line spacing
% Please note that these options may affect formatting.
% Additionally, the use of the \newcommand function should be limited.

\usepackage{lipsum} % Required to insert dummy text
\usepackage[version=4]{mhchem}
\usepackage{siunitx}
\DeclareSIUnit\Molar{M}

%%%%%%%%%%%%%%%%%%%%%%%%%%%%%%%%%%%%%%%%%%%%%%%%%%%%%%%%%%%%
%%% ARTICLE SETUP
%%%%%%%%%%%%%%%%%%%%%%%%%%%%%%%%%%%%%%%%%%%%%%%%%%%%%%%%%%%%
\title{Effects of delayed sequence submission and vaccine development on long-term forecast accuracy of seasonal influenza A/H3N2}

\author[1*]{John Huddleston}
\author[2]{Trevor Bedford}
\affil[1]{Vaccine and Infectious Disease Division, Fred Hutchinson Cancer Center, Seattle, WA, USA}
\affil[2]{Howard Hughes Medical Institute, Seattle, WA, USA}

\corr{jhuddles@fredhutch.org}{JH}

%%%%%%%%%%%%%%%%%%%%%%%%%%%%%%%%%%%%%%%%%%%%%%%%%%%%%%%%%%%%
%%% ARTICLE START
%%%%%%%%%%%%%%%%%%%%%%%%%%%%%%%%%%%%%%%%%%%%%%%%%%%%%%%%%%%%

\begin{document}

\maketitle

\begin{abstract}
TKTK
\end{abstract}

\section{Introduction}

TKTK

\begin{figure}[htb]
\includegraphics[width=\linewidth]{figures/distribution_of_delays_and_horizons}
\caption{Model of submission delays and forecast horizons.}
\label{fig:model_of_delays_and_horizons}
\end{figure}

\section{Results}

\subsection{Reducing forecast horizons and submission delays decreases distances between predicted and observed future populations}

Previously, we trained long-term forecasting models that minimized the earth mover's distance \citep{Rubner1998} between predicted and observed future populations \citep{Huddleston2020}.
Each forecast predicted 12 months into the future from a given initial timepoint to a future timepoint.
Specifically, the forecasts predicted the future frequencies of samples circulating at the initial timepoint based on the estimated fitness of each sample.
We calculated the distance between hemagglutinin (HA) amino acid sequences observed at the initial timepoint and the future timepoint, weighting the distance by the predicted future frequency of samples at the initial timepoint and the observed frequency of samples at the future timepoint.
The best forecasting models used fitness estimates that reduced the distance between predicted and observed future populations.
For example, the most accurate sequence-only model for the 12-month forecast horizon estimated fitness with local branching index (LBI) \citep{Neher:2014eu} and mutational load \citep{Luksza:2014hj}.

To understand the effects of reducing forecast horizons and submission delays on long-term forecast accuracy, we produced forecasts 3, 6, 9, and 12 months into the future using HA sequences available at each initial timepoint based on one of three scenarios of delayed sequence submission to public databases: no delay, an ``ideal'' delay ($\sim$1-month average), and a ``realistic'' delay ($\sim$3-month average) (\FIG{model_of_delays_and_horizons}).
For natural A/H3N2 populations, we used the best sequence-only model, LBI and mutational load, which we previously trained on 12-month forecasts without any submission delay.
For simulated A/H3N2-like populations, we used the observed fitness per sample provided by the simulator.
For each forecast horizon and submission delay type, we calculated the earth mover's distance between the predicted future populations under the given delay scenario and the observed future populations without any delay in sequence availability.
We anticipated that reducing either the forecast horizon or the submission delay would reduce the distance to the future in amino acids (AAs), representing increased accuracy of the forecasting models.

As expected, we found that reducing the forecast horizon from the current standard of 12 months linearly reduced the distance to the future population predicted by the LBI and mutational load model (\FIG{h3n2_distances_to_the_future}).
Under the all three submission delay scenarios, the distance to the future reduced by approximately 1 AA on average for each 3-month reduction in forecast horizon (\TABLE{h3n2_distances_to_the_future}).
We observed the greatest average reduction in distance to the future ($\sim$1.4 AAs) between the 6- and 3-month forecast horizons.
Reducing the forecast horizon also noticeably reduced the variance per timepoint in predicted future populations across all delay scenarios (\FIG{h3n2_distances_to_the_future}).
For example, the standard deviation of distances to the future reduced from $\sim$2.6 AAs at the 12-month horizon to $\sim$1 AA at the 3-month horizon (\TABLE{h3n2_distances_to_the_future}).
We observed the same pattern for simulated A/H3N2-like populations (\FIGSUPP[h3n2_distances_to_the_future]{simulated_distances_to_the_future}).
Thus, reducing how far we have to predict into the future increased both forecast accuracy and precision.

In contrast, we found that reducing submission delays from a $\sim$3-month average delay in the realistic scenario to a $\sim$1-month average delay in the ideal scenario had relatively little effect on distance to the future.
At the 12-month forecast horizon, the ideal and realistic delay scenarios produced similar predictions, with the only noticeable improvement observed under the scenario without any submission delays (\FIG{h3n2_distances_to_the_future}).
As the forecast horizon decreased, the effect of submission delays appeared more prominent, with the greatest effect of reduced delays observed at the 3-month forecast horizon.
However, the average improvement from the realistic to the ideal submission delay scenario at the 3-month horizon was still only $~\sim$0.3 AAs (\TABLE{h3n2_distances_to_the_future}).
Reducing submission delays also had little effect on the variance per timepoint in predicted future populations.
Interestingly, we observed a stronger effect of reducing submission delays in simulated A/H3N2-like populations (\FIGSUPP[h3n2_distances_to_the_future]{simulated_distances_to_the_future}).
We found the best average improvement between realistic and ideal delays of $\sim$0.7 AAs at the 3-month horizon.
As with natural A/H3N2 populations, the effect of reducing submission delays appeared to increase as the forecast horizon decreased.
These results indicate that reducing submission delays may have little effect under the current 12-month forecast approach used for influenza vaccine composition, but reducing submission delays could be increasingly important as we forecast from closer to future influenza populations.

\begin{figure}[htb]
\includegraphics[width=\linewidth]{figures/h3n2_distances_to_the_future_by_delay_and_horizon}
\caption{Distances to the future per timepoint (AAs) for natural A/H3N2 populations by forecast horizon and submission delay type based on forecasts from the local branching index (LBI) and mutational load model.
  Each point represents a future timepoint whose population was predicted from the number of months earlier corresponding to the forecast horizon.
  Points are colored by submission delay type including forecasts made with no delay (blue), an ideal delay (orange), and a realistic delay (green).}
\label{fig:h3n2_distances_to_the_future}
%
\figsupp[Distances to the future for simulated H3N2-like populations]
{Distances to the future per timepoint (AAs) for simulated H3N2-like populations by forecast horizon and submission delay type.
  \figsuppdata{Distances to the future for simulated H3N2-like populations; see \url{https://doi.org/xxx}}}
{\includegraphics[width=6cm]{figures/simulated_distances_to_the_future_by_delay_and_horizon}}\label{figsupp:simulated_distances_to_the_future}
%
\figdata{Distances to the future for natural H3N2 populations.}\label{figdata:h3n2_distances_to_the_future}
\figsrccode{Jupyter notebook used to produce this figure and the supplemental figure lives in \texttt{workflow/notebooks/plot-distances-to-the-future-by-delay-type-and-horizon-for-population.py.ipynb}.}\label{figsrccode:distances_to_the_future}
\end{figure}

\begin{table}[htb]
  \begin{center}
    
\begin{tabular*}{0.7\textwidth}{rrrr}
\toprule
          & \multicolumn{3}{c}{Distance to future (mean +/- std dev AAs)} \\
  Horizon & No delay & Ideal delay & Realistic delay \\
\midrule

3 & 2.91 +/- 0.86 & 3.32 +/- 0.96 & 3.85 +/- 1.05 \\
6 & 4.44 +/- 1.39 & 4.74 +/- 1.54 & 5.03 +/- 1.66 \\
9 & 5.48 +/- 2.05 & 5.84 +/- 2.14 & 6.04 +/- 2.15 \\
12 & 6.45 +/- 2.72 & 6.77 +/- 2.80 & 6.78 +/- 2.61 \\

\bottomrule
\end{tabular*}


    \caption{Distances to the future in amino acids (mean +/- standard deviation AAs) by forecast horizon (in months) and submission delay for H3N2 populations.}
    \label{tab:h3n2_distances_to_the_future}
  \end{center}
\end{table}

\subsection{Reducing submission delays increases accuracy of current clade frequency estimates}

Although the distance between predicted and observed future populations in amino acids provides an unbiased metric to optimize forecasting models, in practice, we use these models to forecast clade frequencies.
To produce these forecasts, we estimate the frequencies of extant clades at an initial timepoint and predict their future frequencies from those initial frequencies and the fitness estimates of each clade's tips.
Given the importance of initial clade frequencies in these forecasts, we tested the effect of submission delays on current clade frequency estimates.
For each timepoint and each clade with an initial frequency of at least 1\%, we calculated the clade frequency error as the difference between clade frequency without submission delays and the frequency with either an ideal or realistic delay.
Positive error values represented underestimation of current clades, while negative values represented overestimation.

We found that our estimates of current clade frequencies were normally distributed with lower variance in the ideal delay scenario than under realistic delays (\FIG{h3n2_current_clade_frequency_errors}A and B).
For small clades ($\ge$1\% and $<$10\% frequency without delays), errors under ideal delays ranged from -3\% to 4\% with a standard deviation of 1\%, while realistic delays produced errors ranging from -8\% to 7\% with a standard deviation of 2\% (\FIG{h3n2_current_clade_frequency_errors}C).
We did not observe a bias toward underestimation or overestimation of initial small clade frequencies under either delay scenario.
For large clades ($\ge$10\% frequency), errors under ideal delay ranged from -6\% to 9\% with a standard deviation of 3\% (\FIG{h3n2_current_clade_frequency_errors}D).
Errors under realistic delays ranged from -17\% to 14\% with a standard deviation of 6\%.
We observed a slight bias toward underestimation of large clades under the realistic delay scenario, with a median error of 1\%.
These results show that reducing submission delays for natural H3N2 populations from a 3-month average to a 1-month average could reduce the standard deviation of current clade frequency errors by half.

\begin{figure}[htb!]
\includegraphics[width=\linewidth]{figures/h3n2_current_frequency_errors_by_delay}
\caption{Clade frequency errors for natural A/H3N2 clades at the same timepoint calculated as the difference between clade frequencies without submission delay and corresponding frequencies with either A) ideal or B) realistic submission delays.
Distributions of frequency errors appear normally distributed in both delay scenarios for both C) small clades ($\ge$1\% and $<$10\% frequency) and D) large clades ($\ge$10\%).
Dashed lines indicate the median error from the distribution of the delay type with the same color.}
\label{fig:h3n2_current_clade_frequency_errors}
%
\figsupp[Current clade frequency errors for simulated H3N2-like populations]
{Clade frequency errors between simulated H3N2-like HA populations with ideal or realistic submission delays and populations without any submission delay.
  \figsuppdata{Current frequencies per tip in each simulated H3N2-like tree of the forecast analysis by delay type (none, ideal, and realistic); see \url{https://doi.org/xxx}}}
{\includegraphics[width=6cm]{figures/simulated_current_frequency_errors_by_delay}}\label{figsupp:simulated_current_clade_frequency_errors}
%
\figdata{Current frequencies per tip in each natural H3N2 tree of the forecast analysis by delay type (none, ideal, and observed).}\label{figdata:h3n2_tip_attributes}
\figsrccode{Jupyter notebook used to produce this figure and the supplemental figure lives in \texttt{workflow/notebooks/plot-current-clade-frequency-errors-by-delay-type-for-populations.py.ipynb}.}\label{figsrccode:current_clade_frequency_errors}
\end{figure}

Delayed submissions similarly affected clade frequencies for simulated A/H3N2-like populations (\FIGSUPP[h3n2_current_clade_frequency_errors]{simulated_current_clade_frequency_errors}).
Small clade errors under ideal delays ranged from -3\% to 5\% (standard deviation of 1\%) and under realistic delays ranged from -8\% to 10\% (standard deviation of 2\%) (\FIGSUPP[h3n2_current_clade_frequency_errors]{simulated_current_clade_frequency_errors}C).
For large clades, errors under ideal delays ranged from -10\% to 11\% (standard deviation of 3\%) and under realistic delays from -15\% to 26\% (standard deviation of 6\%) (\FIGSUPP[h3n2_current_clade_frequency_errors]{simulated_current_clade_frequency_errors}D).
As with natural H3N2 populations, we observed a slight bias in simulated populations under realistic delays toward underestimation of large clade frequencies with a median error of 1\%.
We also observed the same 50\% reduction in standard deviation of current frequency errors for these simulated H3N2-like populations that we saw for natural H3N2 populations.

\subsection{Reducing forecast horizons increases the accuracy and precision of clade frequency forecasts}

Next, we estimated the effects of different submission delays and forecast horizons on the accuracy of clade frequency forecasts.
For each combination of initial timepoint, future timepoint, and delay scenario (\FIG{model_of_delays_and_horizons}B), we calculated initial and predicted future frequencies for all clades present under the given delay and then calculated the corresponding observed future frequencies without delay for clades that descended from the clades present at the initial timepoint.
For all clades with an initial frequency $\ge$10\%, we calculated the error in forecast frequencies as the difference between predicted future frequencies under the given delay scenario and observed future frequencies without any delay.

As with current clade frequency errors, forecast clade frequency errors appeared to be normally distributed (\FIG{h3n2_forecast_clade_frequency_errors}).
This pattern matched our expectation that at any given initial timepoint the overestimation of one clade's future frequency must cause an underestimation of another current clade's future frequency.
The standard deviation of forecast errors decreased with decreasing forecast horizon and, with the exception of the 12-month horizon, with decreasing submission delays (\TABLE{h3n2_forecast_clade_frequency_errors}).
Reducing the forecast horizon had a greater effect on the variation in forecast errors than reducing submission delays.
For example, the standard deviation of forecast errors at the 12-month horizon under realistic submission delays was 0.30, while the standard deviation for the 6-month horizon under realistic delays was 0.17 (\TABLE{h3n2_forecast_clade_frequency_errors}).
In contrast, the standard deviation at the 12-month horizon under ideal submission delays was slightly higher at 0.32.
For all other forecast horizons, reducing the submission delays from realistic to ideal only reduced the standard deviation by 1\%.
We found similar patterns for the absolute values of forecast clade frequency errors where reducing the forecast horizon from 12 months to 6 reduced the average absolute error from 24\% to 15\%, but reducing the submission delays under any forecast horizon had little effect (\TABLE{h3n2_forecast_clade_frequency_errors}).
These results show that the effect size of submission delays on forecast errors is low relative to the effect of the forecast horizon.

\begin{figure}[htb]
\includegraphics[width=\linewidth]{figures/h3n2_forecast_frequency_errors_by_delay_and_horizon}
\caption{Forecast clade frequency errors for natural H3N2 populations by forecast horizon in months and submission delay type (none, ideal, or observed).}
\label{fig:h3n2_forecast_clade_frequency_errors}
%
\figsupp[Forecast clade frequency errors for simulated H3N2-like populations.]
{Forecast clade frequency errors for simulated H3N2-like HA populations by forecast horizon in months and submission delay type (none, ideal, or realistic).
  \figsuppdata{Current, estimated future, and observed future clade frequencies per initial timepoint, forecast horizon, and submission delay type for simulated H3N2-like populations; see \url{https://doi.org/xxx}}}
{\includegraphics[width=6cm]{figures/simulated_forecast_frequency_errors_by_delay_and_horizon}}\label{figsupp:simulated_forecast_clade_frequency_errors}
%
\figsupp[Forecast clade frequency errors for H3N2 populations plotted by initial frequency with delays.]
{Forecast clade frequency errors for H3N2 populations plotted by initial frequency with delay.}
{\includegraphics[width=6cm]{figures/h3n2_forecast_frequency_errors_by_initial_frequency_delay_and_horizon}}\label{figsupp:h3n2_forecast_clade_frequency_errors_by_initial_frequency}
%
\figsupp[Forecast clade frequency errors for H3N2 populations plotted by initial frequency errors with delays.]
{Forecast clade frequency errors for H3N2 populations plotted by initial frequency errors with delay.}
{\includegraphics[width=6cm]{figures/h3n2_forecast_frequency_errors_by_initial_frequency_errors_delay_and_horizon}}\label{figsupp:h3n2_forecast_clade_frequency_errors_by_initial_frequency_errors}
%
\figdata{Current, estimated future, and observed future clade frequencies per initial timepoint, forecast horizon, and submission delay type for H3N2 populations.}\label{figdata:h3n2_clade_frequencies}
\figsrccode{Jupyter notebook used to produce this figure and the supplemental figures lives in \texttt{workflow/notebooks/plot-forecast-clade-frequency-errors-by-delay-type-and-horizon-for-population.py.ipynb}.}\label{figsrccode:forecast_clade_frequency_errors}
\end{figure}

We observed a slight but consistent bias toward the overestimation of future clade frequencies with median errors ranging from -5\% to -10\% across all forecast horizons and submission delays (\FIG{h3n2_forecast_clade_frequency_errors} and \TABLE{h3n2_forecast_clade_frequency_errors}).
Although the average absolute clade frequency error consistently decreased with decreasing forecast horizon, we did not see the same decrease in the average clade frequency except at a forecast horizon of 3 months (\TABLE{h3n2_forecast_clade_frequency_errors}).
We observed the same pattern in simulated H3N2-like populations (\FIGSUPP[h3n2_forecast_clade_frequency_errors]{simulated_forecast_clade_frequency_errors}), suggesting that overestimation of large clade frequencies is a feature of the forecasting framework itself rather than a feature of the different viral populations or fitness metrics.
To understand this pattern, we investigated the relationship between forecast errors and initial clade frequencies across forecast horizons and delays.
We found a weak trend toward overestimation of forecast frequencies for clades with higher initial frequencies under a given delay (\FIGSUPP[h3n2_forecast_clade_frequency_errors]{h3n2_forecast_clade_frequency_errors_by_initial_frequency}).
We also found that clades with overestimated initial frequencies tended to have overestimated forecast frequencies (\FIGSUPP[h3n2_forecast_clade_frequency_errors]{h3n2_forecast_clade_frequency_errors_by_initial_frequency_errors}).
These results suggest that larger clades with overestimated initial frequencies cause the observed bias toward overestimating forecast clade frequencies.
This pattern is also consistent with the limitations of predicting the future frequencies of large clades circulating at a given time when extant smaller clades are likely to grow during the period of the forecast horizon.

\begin{table}[htb]
  \begin{center}
    
\begin{tabular*}{1.0\textwidth}{rrrrrrrrrr}
\toprule
        &            & \multicolumn{5}{c}{Clade frequency error (\%)} & \multicolumn{3}{c}{Absolute frequency error (\%)} \\
Horizon & Delay type & Mean & Median & Std Dev & Min & Max & Mean & Median & Std Dev \\
\midrule

3 & none & 1 & 0 & 9 & -28 & 28 & 7 & 6 & 6 \\
3 & ideal & 1 & -0 & 11 & -32 & 36 & 8 & 6 & 7 \\
3 & realistic & 1 & -0 & 13 & -31 & 50 & 10 & 7 & 9 \\
6 & none & 1 & -0 & 17 & -48 & 45 & 12 & 9 & 11 \\
6 & ideal & 1 & -0 & 19 & -50 & 53 & 13 & 9 & 13 \\
6 & realistic & 1 & -0 & 20 & -52 & 75 & 15 & 12 & 14 \\
9 & none & 0 & -1 & 23 & -66 & 59 & 16 & 10 & 17 \\
9 & ideal & 1 & -1 & 25 & -67 & 58 & 18 & 11 & 18 \\
9 & realistic & 1 & -1 & 26 & -67 & 79 & 19 & 12 & 19 \\
12 & none & 0 & -0 & 30 & -82 & 76 & 20 & 10 & 22 \\
12 & ideal & 1 & -0 & 31 & -80 & 74 & 21 & 9 & 23 \\
12 & realistic & 0 & -0 & 31 & -78 & 78 & 20 & 12 & 23 \\

\bottomrule
\end{tabular*}


    \caption{Errors in clade frequencies between observed and predicted values by forecast horizon (in months) and submission delay for H3N2 clades with an initial frequency $\geq$10\% under the given delay scenario.}
    \label{tab:h3n2_forecast_clade_frequency_errors}
  \end{center}
\end{table}

\subsection{Reduced vaccine development time provides the best improvement in forecast accuracy of available realistic interventions}

Although we have investigated the effects of a range of forecast horizons and submission delays, not all of these scenarios are currently realistic.
The most we can hope to reduce the forecast horizon with current mRNA vaccine technology is from 12 months to 6 months and the most we could reduce submission delays would be from an average of 3 months to 1 month \citep{Grant2023}.
In practice, we wanted to know how much a reduction in forecast horizon or submission delay could improve the accuracy of forecasts to each future timepoint.
To determine the effects of realistic interventions on forecast accuracy, we inspected the reduction in total absolute forecast error per future timepoint associated with improved vaccine development (reducing forecast horizon from 12 months to 6 months), improved genomic surveillance (reducing delays from a 3-month average to 1 month), and the combination of both improvements.
We selected all forecasts with a 12-month horizon and a realistic delay, to represent current forecast conditions.
Then, we selected forecasts for the same future timepoints present in the current conditions for a 6-month horizon and a realistic delay, a 12-month horizon and an ideal delay, and 6-month horizon and an ideal delay.
Since forecasts between different initial and future timepoints could be represented by different clades, we could not compare forecasts for specific clades between interventions.
Instead, we calculated the total absolute clade frequency error per future timepoint under each intervention and calculated the improvement in forecast accuracy as the difference in total error between the current conditions and each intervention.
Positive values represented improved forecast accuracy under a given intervention scenario and negative values represented a reduction in accuracy.

\begin{figure}[htb]
\includegraphics[width=\linewidth]{figures/h3n2_effects_of_realistic_interventions}
\caption{Improvement of clade frequency errors for H3N2 populations between the status quo and realistic interventions.}
\label{fig:h3n2_effects_of_realistic_interventions}
%
\figsupp[Distribution of total absolute clade frequency errors summed across clades per future timepoint for H3N2 populations.]
{Distribution of total absolute clade frequency errors summed across clades per future timepoint for H3N2 populations.\
  We calculated the effects of interventions as the difference between these values per future timepoint under the status quo and specific interventions.
  \figsuppdata{Total absolute clade frequency errors per future timepoint for H3N2 populations; see \url{https://doi.org/xxx}}}
{\includegraphics[width=6cm]{figures/h3n2_total_absolute_forecast_frequency_errors_by_delay_and_horizon}}\label{figsupp:h3n2_total_absolute_clade_frequency_errors}
%
\figsupp[Improvement of clade frequency errors for simulated H3N2-like populations.]
{Improvement of clade frequency errors for simulated H3N2-like populations between the status quo and realistic interventions.
  \figsuppdata{Differences in absolute clade frequency error per future timepoint and clade between the status quo and realistic interventions for simulated H3N2-like populations; see \url{https://doi.org/xxx}}}
{\includegraphics[width=6cm]{figures/simulated_effects_of_realistic_interventions}}\label{figsupp:simulated_effects_of_realistic_interventions}
%
\figsupp[Distribution of total absolute clade frequency errors summed across clades per future timepoint for simulated H3N2-like populations.]
{Distribution of total absolute clade frequency errors summed across clades per future timepoint for simulated H3N2-like populations.\
  \figsuppdata{Total absolute clade frequency errors per future timepoint for simulated H3N2-like populations; see \url{https://doi.org/xxx}}}
{\includegraphics[width=6cm]{figures/simulated_total_absolute_forecast_frequency_errors_by_delay_and_horizon}}\label{figsupp:simulated_total_absolute_clade_frequency_errors}
%
\figdata{Differences in absolute clade frequency error per future timepoint and clade between the status quo and realistic interventions for H3N2 populations.}\label{figdata:h3n2_effects_of_realistic_interventions}
\figsrccode{Jupyter notebook used to produce this figure and the supplemental figure lives in \texttt{workflow/notebooks/plot-forecast-clade-frequency-errors-by-delay-type-and-horizon-for-population.py.ipynb}.}\label{figsrccode:effects_of_realistic_interventions}
\end{figure}

\begin{table}[htb]
  \begin{center}
    
\begin{tabular*}{0.7\textwidth}{rrrr}
\toprule
             & \multicolumn{3}{c}{Improvement in absolute clade frequency error} \\
Intervention & Mean & Median & Std Dev \\
\midrule

improved vaccine & 0.13 & 0.06 & 0.28 \\
improved surveillance & -0.00 & 0.00 & 0.12 \\
improved vaccine and surveillance & 0.15 & 0.07 & 0.29 \\

\bottomrule
\end{tabular*}


    \caption{Improvement in H3N2 forecast accuracy under realistic interventions of improved vaccine development (reducing 12-month to 6-month forecast horizon), improved surveillance (reducing submission delays from 3 months on average to 1 month), or a combination of both interventions.
      We measured improvements from the status quo (12-month forecast horizon and 3-month average submission delay) for clades with an initial frequency of $\geq$10\% as the difference in total absolute clade frequency error per future timepoint and the number and proportion of future timepoint forecasts that improved under the intervention.}
    \label{tab:h3n2_effects_of_realistic_interventions}
  \end{center}
\end{table}

Both interventions with improved vaccine development increased forecast accuracy for the majority of future timepoints (\FIG{h3n2_effects_of_realistic_interventions}, \TABLE{h3n2_effects_of_realistic_interventions}, and \FIGSUPP[h3n2_effects_of_realistic_interventions]{h3n2_total_absolute_clade_frequency_errors}).
In contrast, the intervention that only improved genomic surveillance slightly decreased forecast accuracy by an average of 3\%.
Improving vaccine development alone increased total forecast accuracy by 25\% on average, while the addition of improved genomic surveillance under that 6-month forecast horizon increased total forecast accuracy by 27\% on average.
Based on the distributions of total absolute forecast error per future timepoint, we would expect improved genomic surveillance to provide an even more positive effect on forecast accuracy at a forecast horizon of 3 months (\FIGSUPP[h3n2_effects_of_realistic_interventions]{h3n2_total_absolute_clade_frequency_errors}).
We observed similar effects of interventions in simulated H3N2-like populations except that the average effect of reducing submission delays alone was positive for these populations (\FIGSUPP[h3n2_effects_of_realistic_interventions]{simulated_effects_of_realistic_interventions} and \FIGSUPP[h3n2_effects_of_realistic_interventions]{simulated_total_absolute_clade_frequency_errors}).
Forecasts for simulated H3N2-like populations were generally more accurate than forecasts for natural H3N2 populations, most likely because we knew the true fitness of each simulated sample.
The decrease in average accuracy of natural H3N2 forecasts under the improved genomic surveillance intervention could reflect the bias of the LBI and mutational load fitness metrics compared to the true fitness of each sample.
Based on these results, the single most valuable intervention we could make to improve forecast accuracy would be to reduce the forecast horizon to 6 months or less through more rapid vaccine development.
However, as we reduce the forecast horizon, reducing submission delays should have a greater effect on improving forecast accuracy.

\section{Discussion}

TKTK

\section{Methods and Materials}

\subsection{Estimating and assigning submission delays}

We estimated the delay between sample collection and submission of A/H3N2 hemagglutinin (HA) sequences to the GISAID database \citep{gisaid} by calculating the difference in GISAID-annotated submission date and collection date in days for samples collected between January 1, 2019 and January 1, 2020 and with a submission date prior to October 1, 2020.
We selected this period of time as representative of modern genomic surveillance efforts prior to changes in circulation patterns of influenza caused by the SARS-CoV-2 pandemic.
Of the 104,392 HA sequences in GISAID, 11,222 (11\%) were collected during this period with a mean submission delay of 98 days ($\sim$3 months) and a median delay of 74 days.
Only 11\% of sequences (N=1,210) were submitted within 4 weeks of collection, and only 36\% (N=4,057) were submitted within 8 weeks (\FIG{model_of_delays_and_horizons}A, purple).

We modeled the shape of the observed delay distribution as a gamma distribution using a maximum likelihood fit from SciPy 1.10.1 \citep{scipy}.
With this approach, we estimated a shape parameter of 1.76, a scale parameter of 53.18, and location parameter of 3.98.
The product of these shape and scale values corresponded to a mean delay of 93.76 days (\FIG{model_of_delays_and_horizons}A, green).
To assign realistic submission delays to each sample in our analysis, we randomly sampled from this gamma distribution and calculated a ``realistic submission date'' by adding the sampled delay in days to the observed collection date.

Based on the observed rapid submission of SARS-CoV-2 genomes during the first years of the pandemic, we expected that an achievable ``ideal'' submission delay for seasonal influenza sequences would have a 1-month average delay instead of the observed $\sim$3-month delay from the pre-pandemic period.
We modeled this ideal submission delay distribution by dividing the gamma shape parameter by 3 to get a value of 0.59 and a corresponding mean delay of 31.25 days (\FIG{model_of_delays_and_horizons}A, orange).
This approach effectively shifted the realistic gamma toward zero, while maintaining the relatively longer upper tail of the distribution.
To assign ideal submission delays to each sample in our analysis, we randomly sampled from this modified gamma distribution and added the sampled delay in days to the observed collection date.
Additionally, we required that each sample's ``ideal'' delay be less than or equal to its ``realistic'' delay.

\subsection{Forecasting with different forecast horizons}

We tested the effect of forecasting future influenza populations at forecast horizons of 3, 6, 9, and 12 months (\FIG{model_of_delays_and_horizons}B).
Previously, we produced forecasts every 6 months starting from October 1 and April 1 and predicting 12 months into the future \citep{Huddleston2020}.
To support forecasts in 3-month intervals, we produced annotated time trees for 6 years of HA sequences every 3 months with data available up to the first day of January, April, July, and October.
We produced these trees for each timepoint with three different delay scenarios: no delay, ideal delay, and realistic delay.
For each scenario, we selected sequences for analysis at a given timepoint based on their collection date, ideal submission date, or realistic submission date, respectively.
This experimental design produced forecasts for three delay types at each of the four forecast horizons (e.g., \FIG{model_of_delays_and_horizons}B, blue, green, and orange initial timepoints for the 3-month forecast horizon).

Since submission dates were not available prior to April 2005, our analysis of natural A/H3N2 sequences spanned from April 1, 2005 to October 1, 2019.
To simplify the data required for these analyses, we produced forecasts of natural A/H3N2 populations with our best sequence-only model from our prior work \citep{Huddleston2020}, a composite model based on local branching index (LBI) \citep{Neher:2014eu} and mutational load \citep{Luksza:2014hj}.
For simulated A/H3N2-like populations, we produced forecasts with the ``true fitness'' model that relies on the normalized fitness value of each simulated sample.

Each forecast generated a predicted future frequency per sequence in the initial timepoint's tree.
As in our prior work, we calculated the earth mover's distance \citep{Rubner1998} between the predicted and observed future populations using HA amino acid sequences from initial and future timepoints, predicted future frequencies from the initial timepoint, and observed future frequencies from future timepoint.
For the future timepoint, we used data from the ``no delay'' scenario as our truth set, regardless of the delay scenario for the initial timepoint.
This design allowed us to measure the effect of ideal and realistic submission delays on forecast accuracy relative to a scenario with no delays.

\subsection{Defining clades}

Official clade definitions do not exist for all time periods of our analysis of A/H3N2 populations and do not exist at all for simulated A/H3N2-like populations.
Therefore, we defined ``clades'' for natural and simulated HA trees based on the amino acid haplotypes of HA at 49 previously defined epitope sites \citep{Luksza:2014hj}.
This definition of clades follows from realistic clade definitions that tend to depend on mutations in HA1 at known epitope sites.
Although this automated haplotype-based definition of clades tended to create monophyletic groups, the same haplotype could emerge multiple times independently in different phylogenetic clades.

For each tree in our analysis, we assigned clade labels to each tip and internal node based on their amino acid haplotype at epitope sites.
For each tip, this clade label was deterministic across trees, as the sequence of a tip did not change.
Tips also inherited the clade labels of their ancestral nodes in the tree, allowing us to estimate frequencies of larger clades based on the frequencies of tips in those clades.
However, the inferred sequence at internal nodes varied with the tips present in a given tree.
For example, submission delays could change the tips present in trees representing the same timepoint, resulting in different internal nodes and inferred sequences.
Additionally, the continued accumulation of amino acid mutations between timepoints changed the composition of internal nodes over time, complicating the comparison of large clades between initial and future timepoints of a forecast.

To stably define clades for internal nodes in our analysis, we created a single tree with all HA sequences present in our analysis and without delayed sequence submission.
We inferred ancestral sequences for internal nodes in this tree and assigned clades to each node and tip based on their amino acid haplotypes.
We created a mapping between each tip and the clades it descended from that allowed us to track clade frequencies across 6-year trees built per timepoint in the analysis.

\subsection{Estimating current and future clade frequencies}

We estimated current clade frequencies for each timepoint in our analysis by summing the frequencies of tips in a given timepoint's tree by the clade assigned to each tip.
To inspect the effects of submission delays on clade frequency estimates, we calculated the clade frequency error per timepoint and clade by subtracting the clade frequency estimated with ideal or realistic delayed sequence submission from the corresponding clade frequency without delays.
We compared the effects of submission delays for clades of different sizes by filtering clades by their frequency estimated without delays to small clades ($\ge$1\% and $<$10\%) and large clades ($\ge$10\%).

We estimated future clade frequencies for each combination of delay type, initial timepoint, and future timepoint in the analysis using the following steps.
First, we merged the frequencies per tip at each timepoint with all forecasts made from the same timepoint.
The resulting table included the initial and predicted frequencies for each tip for each combination of initial timepoint, future timepoint, forecast horizon, and submission delay.
For each combination of initial timepoint, future timepoint, and forecast horizon under the scenario with no submission delays, we selected the names and frequencies of all tips with nonzero frequencies at the initial or future timepoints.
Using the tip-to-clade mapping described above, we linked the tips between timepoints based on the most derived (not nested) clades present at both timepoints.
Specifically, we found all clades associated with tips at the initial timepoint, calculated clade frequencies, merged tips for clades with less than 10\% frequency into their parent clades, and assigned each future tip to its most derived ancestral clade that was also present at the initial timepoint.
We calculated the observed future frequency of each clade as the sum of tip frequencies at the future timepoint.
Then, for each delay scenario, we assigned to each tip in the initial timepoint the most derived clade present in the set that linked the initial and future timepoints under no delay.
From these clade assignments, we calculated the estimated initial and predicted future clade frequencies at the initial timepoint.
We left-joined the observed future clade frequencies at the future timepoint with the frequencies under the delay scenario, since submission delays could cause some clades from the scenario without delays to be missing at the initial timepoint.
We set the frequency of missing clades at the initial timepoint to zero.
Finally, we calculated the forecast error as the difference between the observed and estimated future clade frequencies.

\section{Acknowledgments}

TKTK

\nocite{*} % This command displays all refs in the bib file. PLEASE DELETE IT BEFORE YOU SUBMIT YOUR MANUSCRIPT!
\bibliography{delays}

\end{document}
