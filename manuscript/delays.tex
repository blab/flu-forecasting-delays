%%%%%%%%%%%%%%%%%%%%%%%%%%%%%%%%%%%%%%%%%%%%%%%%%%%%%%%%%%%%
%%% ELIFE ARTICLE TEMPLATE
%%%%%%%%%%%%%%%%%%%%%%%%%%%%%%%%%%%%%%%%%%%%%%%%%%%%%%%%%%%%
%%% PREAMBLE
\documentclass[9pt,lineno]{elife}
% Use the onehalfspacing option for 1.5 line spacing
% Use the doublespacing option for 2.0 line spacing
% Please note that these options may affect formatting.
% Additionally, the use of the \newcommand function should be limited.

\usepackage{lipsum} % Required to insert dummy text
\usepackage[version=4]{mhchem}
\usepackage{siunitx}
\DeclareSIUnit\Molar{M}

%%%%%%%%%%%%%%%%%%%%%%%%%%%%%%%%%%%%%%%%%%%%%%%%%%%%%%%%%%%%
%%% ARTICLE SETUP
%%%%%%%%%%%%%%%%%%%%%%%%%%%%%%%%%%%%%%%%%%%%%%%%%%%%%%%%%%%%
\title{Effects of delayed strain submission and vaccine development on long-term forecast accuracy of seasonal influenza A/H3N2}

\author[1*]{John Huddleston}
\author[2]{Trevor Bedford}
\affil[1]{Vaccine and Infectious Disease Division, Fred Hutchinson Cancer Center, Seattle, WA, USA}
\affil[2]{Howard Hughes Medical Institute, Seattle, WA, USA}

\corr{jhuddles@fredhutch.org}{JH}

%%%%%%%%%%%%%%%%%%%%%%%%%%%%%%%%%%%%%%%%%%%%%%%%%%%%%%%%%%%%
%%% ARTICLE START
%%%%%%%%%%%%%%%%%%%%%%%%%%%%%%%%%%%%%%%%%%%%%%%%%%%%%%%%%%%%

\begin{document}

\maketitle

\begin{abstract}
TKTK
\end{abstract}

\section{Introduction}

TKTK

\begin{figure}[htb]
\includegraphics[width=\linewidth]{figures/distribution_of_delays_and_horizons}
\caption{Model of submission delays and forecast horizons.}
\label{fig:model_of_delays_and_horizons}
\end{figure}

\section{Results}

TKTK

\subsection{Distances to the future}

\begin{figure}[htb]
\includegraphics[width=\linewidth]{figures/h3n2_distances_to_the_future_by_delay_and_horizon}
\caption{Distances to the future per timepoint (AAs) for natural H3N2 populations by forecast horizon and submission delay type.}
\label{fig:h3n2_distances_to_the_future}
%
\figsupp[Distances to the future for simulated H3N2-like populations]
{Distances to the future per timepoint (AAs) for simulated H3N2-like populations by forecast horizon and submission delay type.
  \figsuppdata{Distances to the future for simulated H3N2-like populations; see \url{https://doi.org/xxx}}}
{\includegraphics[width=6cm]{figures/simulated_distances_to_the_future_by_delay_and_horizon}}\label{figsupp:simulated_distances_to_the_future}
%
\figdata{Distances to the future for natural H3N2 populations.}\label{figdata:h3n2_distances_to_the_future}
\figsrccode{Jupyter notebook used to produce this figure and the supplemental figure lives in \texttt{workflow/notebooks/plot-distances-to-the-future-by-delay-type-and-horizon-for-population.py.ipynb}.}\label{figsrccode:distances_to_the_future}
\end{figure}

\begin{table}[htb]
  \begin{center}
    
\begin{tabular*}{0.7\textwidth}{rrrr}
\toprule
          & \multicolumn{3}{c}{Distance to future (mean +/- std dev AAs)} \\
  Horizon & No delay & Ideal delay & Realistic delay \\
\midrule

3 & 2.91 +/- 0.86 & 3.32 +/- 0.96 & 3.85 +/- 1.05 \\
6 & 4.44 +/- 1.39 & 4.74 +/- 1.54 & 5.03 +/- 1.66 \\
9 & 5.48 +/- 2.05 & 5.84 +/- 2.14 & 6.04 +/- 2.15 \\
12 & 6.45 +/- 2.72 & 6.77 +/- 2.80 & 6.78 +/- 2.61 \\

\bottomrule
\end{tabular*}


    \caption{Distances to the future in amino acids (mean +/- standard deviation AAs) by forecast horizon (in months) and submission delay for H3N2 populations.}
    \label{table_h3n2_distances_to_the_future}
  \end{center}
\end{table}

\subsection{Current clade frequency errors}

\begin{figure}[htb]
\includegraphics[width=\linewidth]{figures/h3n2_current_frequency_errors_by_delay}
\caption{Clade frequency errors between natural H3N2 populations with ideal or observed submission delays and populations without any submission delay.}
\label{fig:h3n2_current_clade_frequency_errors}
%
\figsupp[Current clade frequency errors for simulated H3N2-like populations]
{Clade frequency errors between simulated H3N2-like HA populations with ideal or realistic submission delays and populations without any submission delay.
  \figsuppdata{Current frequencies per tip in each simulated H3N2-like tree of the forecast analysis by delay type (none, ideal, and realistic); see \url{https://doi.org/xxx}}}
{\includegraphics[width=6cm]{figures/simulated_current_frequency_errors_by_delay}}\label{figsupp:simulated_current_clade_frequency_errors}
%
\figdata{Current frequencies per tip in each natural H3N2 tree of the forecast analysis by delay type (none, ideal, and observed).}\label{figdata:h3n2_tip_attributes}
\figsrccode{Jupyter notebook used to produce this figure and the supplemental figure lives in \texttt{workflow/notebooks/plot-current-clade-frequency-errors-by-delay-type-for-populations.py.ipynb}.}\label{figsrccode:current_clade_frequency_errors}
\end{figure}

\subsection{Forecast clade frequency errors}

\begin{figure}[htb]
\includegraphics[width=\linewidth]{figures/h3n2_forecast_frequency_errors_by_delay_and_horizon}
\caption{Forecast clade frequency errors for natural H3N2 populations by forecast horizon in months and submission delay type (none, ideal, or observed).}
\label{fig:h3n2_forecast_clade_frequency_errors}
%
\figsupp[Absolute forecast clade frequency errors for H3N2 populations.]
{Absolute forecast clade frequency errors for natural H3N2 populations by forecast horizon in months and submission delay type (none, ideal, or observed).
  \figsuppdata{Current, estimated future, and observed future clade frequencies per initial timepoint, forecast horizon, and submission delay type for H3N2 populations; see \url{https://doi.org/xxx}}}
{\includegraphics[width=6cm]{figures/h3n2_absolute_forecast_frequency_errors_by_delay_and_horizon}}\label{figsupp:h3n2_absolute_forecast_clade_frequency_errors}
%
\figsupp[Forecast clade frequency errors for simulated H3N2-like populations.]
{Forecast clade frequency errors for simulated H3N2-like HA populations by forecast horizon in months and submission delay type (none, ideal, or realistic).
  \figsuppdata{Current, estimated future, and observed future clade frequencies per initial timepoint, forecast horizon, and submission delay type for simulated H3N2-like populations; see \url{https://doi.org/xxx}}}
{\includegraphics[width=6cm]{figures/simulated_forecast_frequency_errors_by_delay_and_horizon}}\label{figsupp:simulated_forecast_clade_frequency_errors}
%
\figsupp[Absolute forecast clade frequency errors for simulated H3N2-like populations.]
{Absolute forecast clade frequency errors for simulated H3N2-like HA populations by forecast horizon in months and submission delay type (none, ideal, or realistic).
  \figsuppdata{Current, estimated future, and observed future clade frequencies per initial timepoint, forecast horizon, and submission delay type for simulated H3N2-like HA populations; see \url{https://doi.org/xxx}}}
{\includegraphics[width=6cm]{figures/simulated_absolute_forecast_frequency_errors_by_delay_and_horizon}}\label{figsupp:simulated_absolute_forecast_clade_frequency_errors}
%
\figdata{Current, estimated future, and observed future clade frequencies per initial timepoint, forecast horizon, and submission delay type for H3N2 populations.}\label{figdata:h3n2_clade_frequencies}
\figsrccode{Jupyter notebook used to produce this figure and the supplemental figures lives in \texttt{workflow/notebooks/plot-forecast-clade-frequency-errors-by-delay-type-and-horizon-for-population.py.ipynb}.}\label{figsrccode:forecast_clade_frequency_errors}
\end{figure}

\begin{table}[htb]
  \begin{center}
    
\begin{tabular*}{1.0\textwidth}{rrrrrrrrrr}
\toprule
        &            & \multicolumn{5}{c}{Clade frequency error (\%)} & \multicolumn{3}{c}{Absolute frequency error (\%)} \\
Horizon & Delay type & Mean & Median & Std Dev & Min & Max & Mean & Median & Std Dev \\
\midrule

3 & none & 1 & 0 & 9 & -28 & 28 & 7 & 6 & 6 \\
3 & ideal & 1 & -0 & 11 & -32 & 36 & 8 & 6 & 7 \\
3 & realistic & 1 & -0 & 13 & -31 & 50 & 10 & 7 & 9 \\
6 & none & 1 & -0 & 17 & -48 & 45 & 12 & 9 & 11 \\
6 & ideal & 1 & -0 & 19 & -50 & 53 & 13 & 9 & 13 \\
6 & realistic & 1 & -0 & 20 & -52 & 75 & 15 & 12 & 14 \\
9 & none & 0 & -1 & 23 & -66 & 59 & 16 & 10 & 17 \\
9 & ideal & 1 & -1 & 25 & -67 & 58 & 18 & 11 & 18 \\
9 & realistic & 1 & -1 & 26 & -67 & 79 & 19 & 12 & 19 \\
12 & none & 0 & -0 & 30 & -82 & 76 & 20 & 10 & 22 \\
12 & ideal & 1 & -0 & 31 & -80 & 74 & 21 & 9 & 23 \\
12 & realistic & 0 & -0 & 31 & -78 & 78 & 20 & 12 & 23 \\

\bottomrule
\end{tabular*}


    \caption{Errors in clade frequencies between observed and predicted values by forecast horizon (in months) and submission delay for H3N2 clades with an initial frequency $\geq$10\%.}
    \label{table_h3n2_forecast_frequency_errors}
  \end{center}
\end{table}

\begin{table}[htb]
  \begin{center}
    
\begin{tabular*}{0.7\textwidth}{rrrrr}
\toprule
        &            & \multicolumn{3}{c}{Absolute clade frequency error} \\
Horizon & Delay type & Mean & Median & Std Dev \\
\midrule

3 & none & 0.07 & 0.05 & 0.06 \\
6 & none & 0.11 & 0.08 & 0.10 \\
9 & none & 0.16 & 0.10 & 0.16 \\
12 & none & 0.20 & 0.11 & 0.22 \\
3 & ideal & 0.07 & 0.05 & 0.06 \\
6 & ideal & 0.12 & 0.09 & 0.11 \\
9 & ideal & 0.17 & 0.10 & 0.17 \\
12 & ideal & 0.21 & 0.11 & 0.24 \\
3 & realistic & 0.08 & 0.06 & 0.07 \\
6 & realistic & 0.14 & 0.08 & 0.13 \\
9 & realistic & 0.19 & 0.10 & 0.19 \\
12 & realistic & 0.22 & 0.12 & 0.25 \\

\bottomrule
\end{tabular*}


    \caption{Absolute errors in clade frequencies between observed and predicted values by forecast horizon (in months) and submission delay for H3N2 clades with an initial frequency $\geq$10\%.}
    \label{table_h3n2_absolute_forecast_frequency_errors}
  \end{center}
\end{table}

\subsection{Effects of realistic interventions on forecast accuracy}

\begin{figure}[htb]
\includegraphics[width=\linewidth]{figures/h3n2_effects_of_realistic_interventions}
\caption{Improvement of clade frequency errors for H3N2 populations between the status quo and realistic interventions.}
\label{fig:h3n2_effects_of_realistic_interventions}
%
\figsupp[Improvement of clade frequency errors for simulated H3N2-like populations]
{Improvement of clade frequency errors for simulated H3N2-like populations between the status quo and realistic interventions.
  \figsuppdata{Differences in absolute clade frequency error per future timepoint and clade between the status quo and realistic interventions for simulated H3N2-like populations; see \url{https://doi.org/xxx}}}
{\includegraphics[width=6cm]{figures/simulated_effects_of_realistic_interventions}}\label{figsupp:simulated_effects_of_realistic_interventions}
%
\figdata{Differences in absolute clade frequency error per future timepoint and clade between the status quo and realistic interventions for H3N2 populations.}\label{figdata:h3n2_effects_of_realistic_interventions}
\figsrccode{Jupyter notebook used to produce this figure and the supplemental figure lives in \texttt{workflow/notebooks/plot-forecast-clade-frequency-errors-by-delay-type-and-horizon-for-population.py.ipynb}.}\label{figsrccode:effects_of_realistic_interventions}
\end{figure}

\begin{table}[htb]
  \begin{center}
    
\begin{tabular*}{0.7\textwidth}{rrrr}
\toprule
             & \multicolumn{3}{c}{Improvement in absolute clade frequency error} \\
Intervention & Mean & Median & Std Dev \\
\midrule

improved vaccine & 0.13 & 0.06 & 0.28 \\
improved surveillance & -0.00 & 0.00 & 0.12 \\
improved vaccine and surveillance & 0.15 & 0.07 & 0.29 \\

\bottomrule
\end{tabular*}


    \caption{Improvement in absolute clade frequency errors between the status quo (12-month forecast horizon and realistic submission delay) and realistic interventions for H3N2 clades with an initial frequency of $\geq$10\%.}
    \label{table_h3n2_effects_of_realistic_interventions}
  \end{center}
\end{table}

\section{Discussion}

TKTK

\section{Methods and Materials}

\subsection{Estimating and assigning submission delays}

We estimated the delay between sample collection and submission of A/H3N2 hemagglutinin (HA) sequences to the GISAID database \citep{gisaid} by calculating the difference in GISAID-annotated submission date and collection date in days for samples collected between January 1, 2019 and January 1, 2020 and with a submission date prior to October 1, 2020.
We selected this period of time as representative of modern genomic surveillance efforts prior to changes in circulation patterns of influenza caused by the SARS-CoV-2 pandemic.
Of the 104,392 HA sequences in GISAID, 11,222 (11\%) were collected during this period with a mean submission delay of 98 days ($\sim$3 months) and a median delay of 74 days.
Only 11\% of sequences (N=1,210) were submitted within 4 weeks of collection, and only 36\% (N=4,057) were submitted within 8 weeks (Figure~\ref{fig:model_of_delays_and_horizons}A, purple).

We modeled the shape of the observed delay distribution as a gamma distribution using a maximum likelihood fit from SciPy 1.10.1 \citep{scipy}.
With this approach, we estimated a shape parameter of 1.76, a scale parameter of 53.18, and location parameter of 3.98.
The product of these shape and scale values corresponded to a mean delay of 93.76 days (Figure~\ref{fig:model_of_delays_and_horizons}A, green).
To assign realistic submission delays to each sample in our analysis, we randomly sampled from this gamma distribution and calculated a ``realistic submission date'' by adding the sampled delay in days to the observed collection date.

Based on the observed rapid submission of SARS-CoV-2 genomes during the first years of the pandemic, we expected that an achievable ``ideal'' submission delay for seasonal influenza sequences would have a 1-month average delay instead of the observed $\sim$3-month delay from the pre-pandemic period.
We modeled this ideal submission delay distribution by dividing the gamma shape parameter by 3 to get a value of 0.59 and a corresponding mean delay of 31.25 days (Figure~\ref{fig:model_of_delays_and_horizons}A, orange).
This approach effectively shifted the realistic gamma toward zero, while maintaining the relatively longer upper tail of the distribution.
To assign ideal submission delays to each sample in our analysis, we randomly sampled from this modified gamma distribution and added the sampled delay in days to the observed collection date.
Additionally, we required that each sample's ``ideal'' delay be less than or equal to its ``realistic'' delay.

\section{Acknowledgments}

TKTK

\nocite{*} % This command displays all refs in the bib file. PLEASE DELETE IT BEFORE YOU SUBMIT YOUR MANUSCRIPT!
\bibliography{delays}

\end{document}
